\documentclass[11pt,a4paper]{moderncv}
\moderncvtheme[blue]{classic}
\usepackage[utf8]{inputenc}
\usepackage[top=1.5cm, bottom=1.5cm, left=1.7cm, right=1.7cm]{geometry}
\usepackage[english]{babel}
\setlength{\hintscolumnwidth}{3.5cm} % width of the left column (dates)
\usepackage{multicol} % multi-columns itemize
%\setlength{\makecvtitlenamewidth}{20cm}

\firstname{Romain}
\familyname{Pellerin}
\title{Software Engineer}
%\photo[54pt][0.0pt]{picture} % '64pt' is the height the picture must be resized to, 0.0pt is the thickness of the frame around it (put it to 0pt for no frame) and 'picture' is the name of the picture file
\address{}{\textbf{Open to relocation}}{} % street, city, country
\email{contact@romainpellerin.eu}
\homepage{romainpellerin.eu}
\mobile{+33 6 95 60 57 81}
\extrainfo{\href{https://github.com/rpellerin}{github.com/rpellerin}}
%\quote{Objective: a full-time job from September 2017}
\renewcommand*{\quotefont}{\Large\bfseries} % override quote's default style

% I took the original code from moderncvstylebanking.sty and changed line 25
% USEFUL for theme banking
% \makeatletter
% \renewcommand*{\maketitle}{%
%   \setlength{\maketitlewidth}{1.0\textwidth}%
%   \hfil%
%   \parbox{\maketitlewidth}{%
%     \centering%
%     % name and title
%     \namestyle{\@firstname~\@lastname}%
%     \ifthenelse{\equal{\@title}{}}{}{\titlestyle{~|~\@title}}\\% \isundefined doesn't work on \@title, as LaTeX itself defines \@title (before it possibly gets redefined by \title)
%     % detailed information
%     \addressfont\color{color2}%
%     \ifthenelse{\isundefined{\@addressstreet}}{}{\addtomaketitle{\addresssymbol\@addressstreet}%
%       \ifthenelse{\equal{\@addresscity}{}}{}{\addtomaketitle[~--~]{\@addresscity}}% if \addresstreet is defined, \addresscity and \addresscountry will always be defined but could be empty
%       \ifthenelse{\equal{\@addresscountry}{}}{}{\addtomaketitle[~--~]{\@addresscountry}}%
%       \flushmaketitle\@firstmaketitleelementtrue\\}%
%     \collectionloop{phones}{% the key holds the phone type (=symbol command prefix), the item holds the number
%       \addtomaketitle{\csname\collectionloopkey phonesymbol\endcsname\collectionloopitem}}%
%     \ifthenelse{\isundefined{\@email}}{}{\addtomaketitle{\emailsymbol\emaillink{\@email}}}%
%     \ifthenelse{\isundefined{\@homepage}}{}{\addtomaketitle{\homepagesymbol\httplink{\@homepage}}}%
%     \collectionloop{socials}{% the key holds the social type (=symbol command prefix), the item holds the link
%       \addtomaketitle{\csname\collectionloopkey socialsymbol\endcsname\collectionloopitem}}%
%     \ifthenelse{\isundefined{\@extrainfo}}{}{\addtomaketitle{\@extrainfo}}%
%     \flushmaketitle}\\[2.5em]}% need to force a \par after this to avoid weird spacing bug at the first section if no blank line is left after \maketitle
%  \makeatother

\begin{document}
\makecvtitle
\vspace{-25pt}
\section{Professional Experience}
\cventry{Mar 2017 -- Present}{Software Engineer}{\href{https://www.doctolib.fr/}{Doctolib}}{Paris (France)}{}{
Founded in 2013, Doctolib is the largest ehealth company in Europe. It offers doctors and hospitals a software solution with a full-range of services to improve the efficiency of their organization, provide their patients with a new high-end experience, grow their activity and develop their cooperation with other doctors.\\
\\
As a software engineer, most of my time is dedicated to full-stack development. Our backend is built with Ruby on Rails and PostgreSQL. Our frontend is written in modern JavaScript (React and RxJS). We have over 10,000 end-to-end integration tests, and a few hundred unit tests (both JavaScript and Ruby).\\
\\
Besides code writing, my missions involve:
\begin{itemize}
    \item Leading the frontend committee (decision-making, giving training sessions)
    \item Helping onboard new joiners
    \item Project management on a regular basis (the role frequently rotate among members of each feature team)
    \item Hiring software engineers
\end{itemize}~\\
Other relevant tools, frameworks and SaaS used: Heroku CI, Jenkins, NodeJS, Jest, Capybara, SCSS, Bash.
}
\vspace{10pt}
\cventry{Feb 2017 -- Feb 2018}{Software Engineer}{\href{https://matters.tech/}{Matters}}{San Francisco (USA) and Paris (France)}{}{
\begin{itemize}
    \item As a full time employee from Sep 2017 to Feb 2018 in Paris, I worked for \href{https://wisepops.com}{Wisepops.com} on their new popup editor.\\Stack: JavaScript (ES6+), CSS (displaying popups is quite challenging), React, Redux and NodeJS.
    \item As an intern from Feb 2017 to Jul 2017 in San Francisco, I worked on a brand new project: \href{https://teamstarter.co/}{Teamstarter}. Also experimented new technologies: asm.js and WebAssembly. My missions involved React, Redux, \href{https://sequelizejs.com/}{Sequelize}, \href{https://graphql.org/}{GraphQL}, \href{https://www.apollodata.com/}{Apollo}, PWA (Service Workers), CDNs.
\end{itemize}
}
\vspace{10pt}
\cventry{Sep 2015 -- Feb 2016}{Intern in mobile application development}{\href{http://www.wearesmiths.com/}{The Smiths}}{Amsterdam (The Netherlands)}{}{Developed Android and iOS applications with the Titanium framework.
\begin{itemize}
  \item Front-end with the following frameworks: Backbone, Underscore, Titanium/Alloy
  \item Back-end with Express, hosted on \href{http://www.parse.com/}{parse.com}; also some SQL and Shell scripting
\end{itemize}}
\vspace{10pt}
\cventry{Sep 2014 -- Jun 2015}{Computer Science Manager}{\href{http://www.usec-utc.fr/}{USEC}}{Compiègne (France)}{}{Was in charge of software development projects for local companies.
\begin{itemize}
  \item Wrote official documents (quotes, specifications, customer agreements, etc.)
  \item Recruited and mentored students (those who develop our clients' projects)
\end{itemize}}
\vspace{10pt}
\cventry{Jun 2013 -- Sep 2014}{Android and web developer}{Self-employed and intern}{Nantes (France)}{}{Developed and then updated the Android application and the website of the startup \textsc{\href{http://www.who-wanna.com/en/}{WhoWanna}}. + PHP, MySQL/PostgreSQL, Scala}%\newline{}

\vspace{10pt}
\section{Education}
\cventry{Sep 2014 -- Jul 2017}{Diplôme d'ingénieur en informatique (Software engineering degree)}{\href{https://www.utc.fr/formations-enseignements/genie-informatique.php}{Université de Technologie de Compiègne}}{Compiègne (France)}{}{} % year, degree, institution, city+picture = {\includegraphics[scale=0.5]{picture}City}, grade, description
\vspace{10pt}
\cventry{Sep 2016 -- Dec 2016}{Exchange program}{\href{https://www.kaist.ac.kr/html/en/}{KAIST}}{Daejeon (South Korea)}{}{
\begin{multicols}{2}
\begin{itemize}
\item \textbf{CS453} Automated Software Testing
\item \textbf{CS459} Introduction to Services Computing
\item \textbf{CS540} Network Architecture
\item \textbf{KSE652} Social Computing Systems Design and Analysis
\end{itemize}
\end{multicols}
} % year, degree, institution, city+picture = {\includegraphics[scale=0.5]{picture}City}, grade, description
\vspace{10pt}
\cventry{Sep 2012 -- Jul 2014}{DUT Informatique (Two-year university degree in Computer Science)}{\href{https://www.iutnantes.univ-nantes.fr/321/0/fiche___formation/}{Institut Universitaire de Technologie de Nantes}}{Nantes (France)}{}{}

\vspace{10pt}
\section{Volunteer Experience}
\cventry{Mar 2015 -- Jun 2015}{President and co-organizer of tech talks}{\href{https://www.youtube.com/playlist?list=PL3nMxbEwNq0wE3BNx5b0EEtp8h-yvHyIy}{HumanTalks}}{Compiègne (France)}{}{10-minute talks given once a month about programming languages, tools, software development, etc\\\href{https://humantalks.com/cities/compiegne}{https://humantalks.com/cities/compiegne}}

\vspace{10pt}
\section{Computer Skills}
% \cvcomputer{category}{programs}{category}{programs}
% \cvdoubleitem{subtitle}{text}{subtitle}{text}
\cvitem{Languages}{\textbf{JavaScript (ES6+), HTML, CSS}, Python, Bash, PHP, Java, C++ (Qt)}
\cvitem{Databases}{\textbf{PostgreSQL, MySQL}, SQLite (+ UML as a modeling language)}
\cvitem{OSes}{\textbf{GNU/Linux} (\href{https://xubuntu.org/}{Xubuntu} user, a Debian/Ubuntu-based distro)}
\cvitem{Other}{\textbf{React, Git}, Apache2, system administration (iptables, fail2ban), OpenVPN, TCP/IP, \LaTeX}

\vspace{10pt}
\section{Languages}
\cvlanguage{French}{Native speaker}{}
\cvlanguage{English}{Full professional proficiency -- European C1/C2 level}{\textbf{TOEIC 985/990 (March 2016)}}

\vspace{10pt}
\section{Hobbies}
\cvdoubleitem{\textbf{Raspberry Pi:}}{Websites, VPN, Nextcloud}{\textbf{Side-projects:}}{Android apps, web browser add-ons}
\cvdoubleitem{\textbf{Conferences:}}{I gave talks and attended some, such as \href{https://devfest.gdgnantes.com/}{DevFest} or \href{https://web2day.co/}{Web2Day}}{\textbf{Free software:}}{Supporter; most of my source code is on \href{https://github.com/rpellerin}{GitHub}}
\cvdoubleitem{\textbf{Music:}}{Guitar for fifteen years}{\textbf{Sport:}}{Workout}
\end{document}
