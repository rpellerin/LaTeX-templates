\documentclass[11pt,a4paper]{moderncv}
\moderncvtheme[blue]{classic}
\nopagenumbers
\usepackage[utf8]{inputenc}
\usepackage[top=1.1cm, bottom=1.1cm, left=1.0cm, right=1.0cm]{geometry}
\usepackage[english]{babel}
\setlength{\hintscolumnwidth}{3.5cm} % width of the left column (dates)
\usepackage{multicol} % multi-columns itemize
%\setlength{\makecvtitlenamewidth}{20cm}

\firstname{Romain}
\familyname{Pellerin}
\title{Engineering Manager at Doctolib GmbH}
%\photo[54pt][0.0pt]{picture} % '64pt' is the height the picture must be resized to, 0.0pt is the thickness of the frame around it (put it to 0pt for no frame) and 'picture' is the name of the picture file
\address{}{Berlin, Germany}{} % street, city, country
%\email{}
\homepage{romainpellerin.eu}
\social[github]{rpellerin}
%\mobile{+33 6 95 60 57 81}
\extrainfo{CV updated on \today}
%\quote{Objective: a full-time job from September 2017}
\renewcommand*{\quotefont}{\Large\bfseries} % override quote's default style
\definecolor{color1}{rgb}{0,0,0} % Overwrites the default blue color

\renewcommand*{\labelitemi}{--}
% I took the original code from moderncvstylebanking.sty and changed line 25
% USEFUL for theme banking
% \makeatletter
% \renewcommand*{\maketitle}{%
%   \setlength{\maketitlewidth}{1.0\textwidth}%
%   \hfil%
%   \parbox{\maketitlewidth}{%
%     \centering%
%     % name and title
%     \namestyle{\@firstname~\@lastname}%
%     \ifthenelse{\equal{\@title}{}}{}{\titlestyle{~|~\@title}}\\% \isundefined doesn't work on \@title, as LaTeX itself defines \@title (before it possibly gets redefined by \title)
%     % detailed information
%     \addressfont\color{color2}%
%     \ifthenelse{\isundefined{\@addressstreet}}{}{\addtomaketitle{\addresssymbol\@addressstreet}%
%       \ifthenelse{\equal{\@addresscity}{}}{}{\addtomaketitle[~--~]{\@addresscity}}% if \addresstreet is defined, \addresscity and \addresscountry will always be defined but could be empty
%       \ifthenelse{\equal{\@addresscountry}{}}{}{\addtomaketitle[~--~]{\@addresscountry}}%
%       \flushmaketitle\@firstmaketitleelementtrue\\}%
%     \collectionloop{phones}{% the key holds the phone type (=symbol command prefix), the item holds the number
%       \addtomaketitle{\csname\collectionloopkey phonesymbol\endcsname\collectionloopitem}}%
%     \ifthenelse{\isundefined{\@email}}{}{\addtomaketitle{\emailsymbol\emaillink{\@email}}}%
%     \ifthenelse{\isundefined{\@homepage}}{}{\addtomaketitle{\homepagesymbol\httplink{\@homepage}}}%
%     \collectionloop{socials}{% the key holds the social type (=symbol command prefix), the item holds the link
%       \addtomaketitle{\csname\collectionloopkey socialsymbol\endcsname\collectionloopitem}}%
%     \ifthenelse{\isundefined{\@extrainfo}}{}{\addtomaketitle{\@extrainfo}}%
%     \flushmaketitle}\\[2.5em]}% need to force a \par after this to avoid weird spacing bug at the first section if no blank line is left after \maketitle
%  \makeatother

\begin{document}
\makecvtitle
\vspace{-15pt}
\section{Experience}
\cventry{Sep 2022 -- Present}{Engineering Manager}{\href{https://www.doctolib.de/}{Doctolib GmbH}}{Berlin (Germany)}{}{Managing a team of 6 developers focused on performance and loading time optimization.\\In 2023 I also had another team of 6 developers reporting to me (12 direct reports in total).\\\\
\textbf{Notable achievements:}
\begin{itemize}
    \item 2023: managed the team that played the biggest role in the \href{https://about.doctolib.fr/10-ans-a-vos-cotes/}{10-year anniversary celebration and rebranding of the company} - coordinated the first ever release cycle of the company for this event
\end{itemize}
}
\vspace{11pt}
\cventry{Oct 2020 -- Sep 2022}{Staff Software Engineer}{\href{https://www.doctolib.de/}{Doctolib GmbH}}{Berlin (Germany)}{}{
\begin{itemize}
    \item Support the feature teams that are working on the website for healthcare professionals: review of technical pre-project documents, pair programming, advise and help regarding technical debt and legacy code, "tech tasks" priorization and reporting
    \item Cross-domain work: contribution to large-scale projects, refactoring, mentorship (internal program based on weekly 1-on-1 meetings) and trainings (mostly on front-end topics and Git)
    \item Risk management: support, mitigation and prevention of organizational threats and down incidents
    \item Interviewing for engineering director, manager and software engineer roles (on average one interview per week)
    \item Help onboard new joiners and train current employees through an internal ramp-up program
\end{itemize}\leavevmode\\
\textbf{Notable achievements:}
\begin{itemize}
    \item 2018-2023: third all-time biggest code contributor, out of 600+ contributors since the creation of the company
    \item 2019-2022: \href{https://romainpellerin.eu/waving-goodbye-to-internet-explorer-11-in-2021.html}{decided the end of the support of Internet Explorer 11 and other outdated browsers, and led a company-wide program to help Doctolib's customers upgrade their browsers}, from small practices to large hospitals, allowing 1. the removal of tests on these browsers 2. simplifying our front-end build pipeline 3. decreasing the number of incidents related to old browsers close to zero
    \item 2021: made English the default language of the monolith in place of French, enabling 150 international developers in 3 countries to develop and write tests faster
    \item 2018-2021: \href{https://romainpellerin.eu/regaining-control-over-doctolib-com-frontend.html}{led a cross-team effort to refactor more than 600 files written in RxJS to vanilla JavaScript or React hooks}, hence removing the number one pain point that was reported by developers about the front-end stack at that time
    \item 2020: removed Rails Sprockets in favor of Webpack, simplifying our front-end build pipeline and allowing developers to write modern JavaScript everywhere in the codebase
\end{itemize}~\\
\textbf{Stack:} Ruby on Rails, PostgreSQL, Redis, ElasticSearch, React, SASS/SCSS, Docker.\\Our CI pipeline executes 70,000+ integration (Minitest + Capybara), model and unit tests (JavaScript and Ruby).\\Doctolib runs on a huge monolith, made of Rails Engines and a few other standalone services.\\\\
Additional relevant keywords: AWS, New Relic, Datadog, Sentry, SCRUM/agile development, React testing library, Jest, React Query.
}
\vspace{11pt}
\cventry{Jan 2020 -- Oct 2020}{Software Engineer}{\href{https://www.doctolib.de/}{Doctolib GmbH}}{Berlin (Germany)}{}{}
\vspace{11pt}
\cventry{Mar 2018 -- Dec 2019}{Software Engineer}{\href{https://www.doctolib.fr/}{Doctolib}}{Paris (France)}{}{}
\vspace{11pt}
\cventry{Sep 2017 -- Feb 2018}{Software Engineer}{\href{https://matters.tech/}{Matters}}{Paris (France)}{}{
Worked for a client (\textsc{\href{https://wisepops.com}{Wisepops.com}}) on a web-based popup editor built with React.\\
\textbf{Stack:} JavaScript (React, Redux, NodeJS), CSS.
}
\vspace{11pt}
\cventry{Feb 2017 -- Jul 2017}{Software Engineer Intern}{\href{https://matters.tech/}{Matters}}{San Francisco (USA)}{}{
Bootstrapped \href{https://teamstarter.co/}{Teamstarter.co}. Also experimented new technologies: asm.js and WebAssembly.\\
\textbf{Stack:} JavaScript (React, Redux, \href{https://sequelizejs.com/}{Sequelize}), \href{https://graphql.org/}{GraphQL}, \href{https://www.apollodata.com/}{Apollo}, PWA (Service Workers).
}
\vspace{11pt}
\cventry{Sep 2015 -- Feb 2016}{Mobile Developer Intern}{\href{http://www.wearesmiths.com/}{The Smiths}}{Amsterdam (The Netherlands)}{}{
%Mobile and backend development. Also some SQL and Shell-scripting.\\
\textbf{Stack:} \href{https://www.appcelerator.com/Titanium/}{Titanium/Alloy (JavaScript framework)}, Backbone.js, Underscore.js, Express.js, SQL.
}
\vspace{11pt}
\cventry{Jun 2013 -- Sep 2014}{Android and Web Developer}{Freelance}
%and then intern
{Nantes (France)}{}{%Developed and maintained the Android application and the website of the startup \textsc{\href{http://www.who-wanna.com/en/}{WhoWanna}}.\\
\textbf{Stack:} Java, PHP, MySQL, JavaScript, HTML, CSS, Scala.
}
\vspace{11pt}
%\pagebreak
\section{Education}
\cventry{Sep 2014 -- Jul 2017}{Diplôme d'ingénieur en informatique (Software engineering degree)}{\href{https://www.utc.fr/formations-enseignements/genie-informatique.php}{Université de Technologie de Compiègne (UTC)}}{Compiègne (France)}{}{
\begin{multicols}{2}
\begin{itemize}
\item \textbf{LO17} Indexing and search engines
\item \textbf{L021} Object-oriented programming (C++)
\item \textbf{MB11} Revision of mathematics
\item \textbf{MI01} Architecture of a CPU
\item \textbf{NF11} Programming language theory
\item \textbf{NF16} Algorithms and data structures
\item \textbf{NF17} Databases design
\item \textbf{RO03} Combinatorial optimization
\item \textbf{SC12} Technology, cognition, perception
\item \textbf{SI28} Interactive writing and multimedia
\item \textbf{SR02} Operating systems
\item \textbf{SR03} Web application architectures
\item \textbf{SR05} Distributed algorithms and systems
\item \textbf{SY31} Sensors for smart systems
\item \textbf{LA13, LA15, LB14, SI14} English courses
\end{itemize}
\end{multicols}
} % year, degree, institution, city+picture = {\includegraphics[scale=0.5]{picture}City}, grade, description
\vspace{11pt}
\cventry{Sep 2016 -- Dec 2016}{Exchange program}{\href{https://www.kaist.ac.kr/html/en/}{KAIST}}{Daejeon (South Korea)}{}{
\begin{multicols}{2}
\begin{itemize}
\item \textbf{CS453} Automated Software Testing
\item \textbf{CS459} Introduction to Services Computing
\item \textbf{CS540} Network Architecture
\item \textbf{KSE652} Social Computing Systems Design and Analysis
\end{itemize}
\end{multicols}
} % year, degree, institution, city+picture = {\includegraphics[scale=0.5]{picture}City}, grade, description
\vspace{11pt}
\cventry{Sep 2012 -- Jul 2014}{DUT Informatique (Two-year university degree in Computer Science)}{\href{https://www.iutnantes.univ-nantes.fr/321/0/fiche___formation/}{Institut Universitaire de Technologie de Nantes}}{Nantes (France)}{}{
\begin{multicols}{2}
\begin{itemize}
\item Maths (algebra, set theory, graphs, probability and statistics, automata theory)
\item Systems (Unix, x86 assembly, C, security)
\item Networks (IP/TCP, Wifi and Ethernet, HTTP)
\item Object-oriented programming (Java) and design patterns
\item Web (HTML, CSS, JavaScript, PHP)
\item Databases (Oracle, MySQL)
\item Communication
\item Economy
\item Business management
\item English
\end{itemize}
\end{multicols}
}
\vspace{11pt}
\section{Volunteering}
\cventry{Mar 2015 -- Jun 2015}{President and co-organizer of tech talks}{\href{https://www.youtube.com/playlist?list=PL3nMxbEwNq0wE3BNx5b0EEtp8h-yvHyIy}{HumanTalks}}{Compiègne (France)}{}{10-min talks given once a month about programming languages, tools, software development, etc.\\\href{https://humantalks.com/cities/compiegne}{https://humantalks.com/cities/compiegne}}
\cventry{Sep 2014 -- Jun 2015}{Project manager}{\href{http://www.usec-utc.fr/}{USEC} (student society)}{Compiègne (France)}{}{Was in charge of software development projects for local businesses.
\begin{itemize}
  \item Wrote official documents (quotes, specifications, customer agreements)
  \item Recruited and mentored students (those who developed our clients' projects)
\end{itemize}}

\vspace{11pt}
\section{Computer skills}
% \cvcomputer{category}{programs}{category}{programs}
% \cvdoubleitem{subtitle}{text}{subtitle}{text}
\cvitem{Languages}{\textbf{JavaScript, Ruby, HTML, CSS}, Python, PHP, Java, Shell-scripting}
\cvitem{Databases}{\textbf{PostgreSQL, MySQL}, SQLite, Redis, UML}
\cvitem{Operating systems}{\textbf{GNU/Linux} (I am a \href{https://xubuntu.org/}{Xubuntu} user -- a Debian/Ubuntu-based distro)}
\cvitem{Miscellaneous}{\textbf{React, Ruby on Rails, advanced knowledge of Git}, NodeJS, \LaTeX}

\vspace{11pt}
\section{Languages}
\cvlanguage{French}{Native speaker}{}
\cvlanguage{English}{Proficient -- European (CEFR) C2 level (estimated)}{\textbf{TOEIC 985/990 (March 2016)}}
\cvlanguage{German}{Intermediate -- European (CEFR) B1 level (estimated)}{}
\cvlanguage{Spanish}{Beginner -- European (CEFR) A1 level (estimated)}{}

\vspace{11pt}
\section{Hobbies and interests}
\cvdoubleitem{\textbf{Raspberry Pi:}}{Tinkering with Arduino, Nextcloud, cameras, Python, sensors...}{\textbf{Reading:}}{Big fan of comic books (The Walking Dead, Paper Girls, Saga, Low), sci-fi (read Hyperion in 2023), business (The Culture Map is on my nightstand)}
\cvdoubleitem{\textbf{Sport:}}{Road cycling (6,100 kms in 2023), running and weight lifting}{\textbf{Music:}}{Guitar for fifteen years}
\end{document}
